%%%%%%%%%%%%%%%%%%%%%%%%%%%%%%%%%%%%%%%%%
% Beamer Presentation
% LaTeX Template
% Version 1.0 (10/11/12)
%
% This template has been downloaded from:
% http://www.LaTeXTemplates.com
%
% License:
% CC BY-NC-SA 3.0 (http://creativecommons.org/licenses/by-nc-sa/3.0/)
%
%%%%%%%%%%%%%%%%%%%%%%%%%%%%%%%%%%%%%%%%%

%----------------------------------------------------------------------------------------
%	PACKAGES AND THEMES
%----------------------------------------------------------------------------------------

\documentclass{beamer}

\mode<presentation> {

%\usetheme{default}
%\usetheme{AnnArbor}
%\usetheme{Antibes}
%\usetheme{Bergen}
%\usetheme{Berkeley}
%\usetheme{Berlin}
%\usetheme{Boadilla}
%\usetheme{CambridgeUS}
%\usetheme{Copenhagen}
%\usetheme{Darmstadt}
%\usetheme{Dresden}
%\usetheme{Frankfurt}
%\usetheme{Goettingen}
%\usetheme{Hannover}
%\usetheme{Ilmenau}
%\usetheme{JuanLesPins}
%\usetheme{Luebeck}
\usetheme{Madrid}
%\usetheme{Malmoe}
%\usetheme{Marburg}
%\usetheme{Montpellier}
%\usetheme{PaloAlto}
%\usetheme{Pittsburgh}
%\usetheme{Rochester}
%\usetheme{Singapore}
%\usetheme{Szeged}
%\usetheme{Warsaw}

%\usecolortheme{albatross}
%\usecolortheme{beaver}
%\usecolortheme{beetle}
%\usecolortheme{crane}
%\usecolortheme{dolphin}
%\usecolortheme{dove}
%\usecolortheme{fly}
%\usecolortheme{lily}
%\usecolortheme{orchid}
%\usecolortheme{rose}
%\usecolortheme{seagull}
%\usecolortheme{seahorse}
\usecolortheme{whale}
%\usecolortheme{wolverine}

%\setbeamertemplate{footline} % To remove the footer line in all slides uncomment this line
%\setbeamertemplate{footline}[page number] % To replace the footer line in all slides with a simple slide count uncomment this line

%\setbeamertemplate{navigation symbols}{} % To remove the navigation symbols from the bottom of all slides uncomment this line
}

\usepackage{float}
%\usepackage{dcolumn}                  % align numbers by decimal point
\usepackage{pdflscape}                 % Page rotation
\usepackage{fancyvrb}                  % Verbatim'ui
\usepackage{dsfont}
\usepackage[utf8]{inputenc}            % naudojama, kai .tex failas UTF-8 koduotės
\usepackage[L7x]{fontenc}              % nurodoma lietuviško teksto koduotė Latin-7
\usepackage[lithuanian]{babel}         % nurodoma, kad dokumentas yra lietuviškas
\usepackage{verbatim} 
\usepackage{lmodern}                   % dokumente naudojamas šriftas Latin Modern
\usepackage{microtype}                 % optimizuojami atstumai tarp raidžių žodyje
\usepackage{indentfirst}               % atitraukiama pirmoji naujo skyriaus eilutė
\usepackage{icomma}                    % po kablelio skaičiaus viduryje nebus tarpo
\usepackage{amsmath, amssymb, amsthm}  % matematiniai simboliai ir konstrukcijos
\usepackage{graphicx}                  % grafinių failų įterpimas ir kiti nustatymai
\usepackage{xcolor}                    % naudojamas teksto spalvoms reguliuoti
\usepackage{booktabs}                  % reikalingas tvarkingoms lentelėms sudaryti 
                                       % galima pridėt , stackengine
\usepackage{Verbatim}                                        
\usepackage{enumerate}                                       
\usepackage{multirow}                  % paketas kelių eilučių apjungimui lentelėse
\usepackage{array}                     % paketas papildomiems lentelių nustatymams
\usepackage{listings}                  % programinio kodo įterpimas ir formatavimas
\usepackage{caption}                   % paveiksliukų ir lentelių užrašų formavimas
\usepackage{geometry}                  % paraščių ir kitų lapo parametrų nustatymai
\usepackage{hyperref}                  % interaktyvioms nuorodoms dokumente sukurti
\usepackage{amsmath}
\usepackage[final]{pdfpages}           % galima pridėti kitus pdf puslapius
\delimitershortfall-1sp
\newcommand\abs[1]{\left|#1\right|}
\def\UrlBreaks{\do/\do-\do_\do.\do\%} %url breaks at / - _
%\usepackage[nottoc,numbib]{tocbibind}

%\setstackEOL{\#}                        % prie stackengine
%\setstackgap{L}{12pt}                   % prie stackengine
\newcommand*{\Perm}[2]{{}^{#1}\!P_{#2}}  % reikalinga kombinatorikai
\newcommand*{\Comb}[2]{{}^{#1}C_{#2}}%
%\newcolumntype{d}[1]{D{.}{.}{#1}}       % dcolumn package'ui


\captionsetup{format = hang,           % paketo caption parametrų nustatymai
           labelfont = bf,
           tablename = table,
          figurename = fig,  
            labelsep = period
}

\hypersetup{ unicode = true,           % paketo hyperref parametrų nustatymai
         linktocpage = false, 
          colorlinks = true, 
           linkcolor = red,
           citecolor = blue
}

\lstset{       frame = l,              % paketo listings parametrų nustatymai
            framesep = 0.5\parindent,
         xleftmargin = 0.5\parindent,
           linewidth = 1.0\linewidth,
           aboveskip = 0.8\baselineskip,
           belowskip = 0.8\baselineskip,
          keepspaces = true,
          basicstyle = \footnotesize\ttfamily\color{ruda},
           basewidth = 0.5em,
        escapeinside = {(*}{*)},
       extendedchars = \true,
       inputencoding = utf8
}

%----------------------------------------------------------------------------------------
%	TITLE PAGE
%----------------------------------------------------------------------------------------

\institute[]{
% Vilniaus Gedimino technikos universitetas
Vilnius Gediminas Technical University
}
\title[]
{\centering  
Statistical Analysis of Sentence Structure in Lithuanian Texts
% Lietuvių kalbos sakinių struktūros statistinė analizė
}
\author[Neringa Bružaitė, Tomas Rekašius]
{ 
 Neringa Bružaitė \bigskip \\  \small 
 Academic Supervisor: Dr. Tomas Rekašius \vspace{5mm}
}

\date{ \tiny 2017 04 24}



\begin{document}

\begin{frame}
\titlepage % Print the title page as the first slide
\end{frame}

\begin{frame}
\frametitle{Content} 
% Table of contents slide, comment this block out to remove it
\tableofcontents 
% Throughout your presentation, if you choose to use 
% \section{} and \subsection{} commands, these will automatically be printed on 
% this slide as an overview of your presentation
\end{frame}

%----------------------------------------------------------------------------
%	PRESENTATION SLIDES
%----------------------------------------------------------------------------

\section{Object of Research} 
% Sections can be created in order to organize your presentation into 
% discrete blocks, all sections and subsections are automatically printed in 
% the table of contents as an overview of the talk


%--------------------- 1: Object of Research ----------------------------------

\begin{frame}[fragile]
\frametitle{Object of research}
%
Examined \alert{92} text files from morphologically annotated corpus MATAS.
\smallskip
Corpus contains \alert{1641263} words and \alert{138123} sentences.
%
\bigskip

Example of morphologically annotated corpus:
%
\begin{Verbatim}[frame=single, fontsize=\small]
<word="Pasaulio" lemma="pasaulis" type="dktv vyr.gim vnsk K">
<space>
<word="pabaiga" lemma="pabaiga" type="dktv mot.gim vnsk V">
<sep=":">
<space>
<word="apsakymai" lemma="apsakymas" type="dktv vyr.gim dgsk V">
\end{Verbatim}
%
\end{frame}

% --------------------- 2: Coding Sentences-------------------------------------

\section{Coding sentences}
%
\begin{frame}
%
\frametitle{Coding Sentences}
Texts from corpus are coded keeping order of sentences:
\begin{enumerate}[I]
\item -- in the sentences remain only nouns (``D'') and verbs (``V''), and any 
other part of speech are replaced by the symbol ``-''. Several consecutive ``-'' 
symbols are combined;
\item --  obtained from the code of type I, by joining several consecutive nouns 
  (``D'') or verbs (``V'').
\end{enumerate}
%
\begin{table}[h!]
\centering
\begin{tabular}{ll}
\toprule
Structure of sentence & DDDDBD  \\
I encoding            & DDDD-D  \\ 
II encoding           & D-D     \\ 
\bottomrule
\end{tabular}
\end{table}
%
Code of sentence „Lietuvių kalbos sakinių struktūros statistinė analizė“ for 
different encodings. Symbol ``B'' stands for adjective.
%
 \end{frame}
 
 
% -------------------3: Repetition frequencies of codes------------------------

\begin{frame}
\frametitle{Repetition frequencies of codes}
%
In the table below are showed counts of structure codes with various frequencies.
%
\begin{table}[h!]
\centering
% Frequency table, which shows repetition frequencies of counts of codes
\begin{tabular}{r|ccccccccccc}
\toprule
Frequency   & 1   & 2  & 3  & 4  & 5  & 6 & 7  & \dots & 149 & 161 & 196 \\ 
\midrule
I encoding  & 954 & 77 & 34 & 17 & 13 & 2 & 11 & \dots & 0 & 0 & 1 \\ 
II encoding & 632 & 86 & 26 & 18 & 14 & 8 & 7  & \dots & 1 & 1 & 1\\ 
\bottomrule
\end{tabular}
\end{table}
%
\bigskip
%
For codes following rule applies: there are few very high frequency codes, 
and many low frequency codes. This distribution approximately follows 
mathematical form known as \alert{Zipf's law}.
%
\end{frame}

% ------------------- 4: History of Zipf’s law --------------------------------


\begin{frame}
\frametitle{History of Zipf's law}

\begin{itemize}
\item George Zipf was not the first one who noticed such phenomena as the unequal 
distribution of words in the text.
\item Jean-Baptiste Estoup was the first person who noticed and 
mathematically formulated this law in his book \textit{Gammes 
sténographiques} (3d ed. 1912).
\item Edward Condon was the first person, who graphed the Zipf's law (which 
then was not called Zipf's law) in 1928. 
\end{itemize}
% \bigskip
% Zipf's law is also found in:
% \begin{itemize} 
% \item city populations;
% \item solar flare intensities;
% \item earthquake magnitudes and etc.
% \end{itemize}
% \bigskip
% We will check if Zipf's law is valid for the codes of sentence structure formed by I and II encodings.
\end{frame}


% ------------------------- 5: Zipf’s law -------------------------------------

\section{Zipf's law for the structure of sentences}
%
\begin{frame}
\frametitle{Zipf's law}
%
\begin{block}{Zipf's law (1949 m.)}
Empirical law, which describes relation between rank of frequency and 
frequency of a word $f_z$, which has rank $z$
%
\begin{equation} \label{f. zipf_nelog}
f_z = \dfrac{C}{z^\alpha},
\end{equation}
here $z$ -- rank, $\alpha > 0$, $C$ -- const.
%
\end{block}
%
%
\begin{block}{}
Taking logarithms at both sides, the linear relation between 
$\log f_z$  and $\log z$ follows immediately:
%
\begin{equation}
\log f_z = \log C - \alpha\log z.
\end{equation}
%
\end{block}
%
%
\end{frame}


% --------------- 6: Parameter estimates of Zipf’s law ------------------------
\begin{frame}
%
\frametitle{Parameter estimates of Zipf's law}
%
\only<1>
{
 \begin{figure}[h!]
 \centering
 \includegraphics[scale=0.4]{zipfo_param_kodams}
 \end{figure}
}
%
\only<2>
{
 \begin{table}[]
 \centering
 \begin{tabular}{lcc}
 \toprule
         & TRUE & FALSE    \\ \midrule
 $\alpha_I > \alpha_{II}$ & 91        & 1 \\ 
 $\log C_I > \log C_{II}$ & 1        & 91      \\ 
 \bottomrule
 \end{tabular}
 \end{table}
 %
 \bigskip
 %
In the table above $\alpha_I$ and $\log C_I$ are Zipf's law parameters for 
I encoding sentences and $\alpha_{II}$ and $\log C_{II}$ -- for II encoding 
sentences.\\
%
\bigskip
%
There is one text, for which Zipf's parameter estimates are the same for both encodings.
}
%
\end{frame}

% --------- 7: Log-log graphs of sentence structure code frequency ------------

\begin{frame}
\frametitle{Log-log graphs of sentence structure code frequency}
%
\begin{figure}[h!]
\centering
\includegraphics[scale=0.359]{mkm_kodams.pdf}
\end{figure}
%
Code frequency $f_z$ as a function of Zipf rank $z$ in the log-log plane for 
one fiction text, which has $N = 2886$ sentences.
%
\end{frame}

% -------------------- 8: Zipf’s law: predicted frequencies -------------------


\begin{frame}
\frametitle{Zipf's law: predicted frequencies}
%
\only<1>
{
 \begin{figure}[h!]
 \centering
 \includegraphics[scale=0.359]{mkm_visiems10.pdf}
 \end{figure}
}
%
\only<2>
{
 \begin{figure}[h!]
 \centering
 \includegraphics[scale=0.359]{mkm_visiems12.pdf}
 \end{figure}
}
%
\only<3>
{
 \begin{figure}[h!]
 \centering
 \includegraphics[scale=0.359]{mkm_visiems13.pdf}
 \end{figure}
}
%
\only<4>
{
 \begin{figure}[h!]
 \centering
 \includegraphics[scale=0.359]{mkm_visiems14.pdf}
 \end{figure}
}
\end{frame}

% ------------------------ 9: Conclusions and results ----------------------

\begin{frame}
\frametitle{Conclusions and results}
\begin{itemize}
\item A large part of encoded sentences has a standard structure. On the 
other hand, there are a lot of encoded sentences which are unique.
\item The encoded sentences are described by Zipf's law quite well.
\end{itemize}
\end{frame}

%----------------------------------------------------------------------------
%	CONFERENCE MATERIALS
%----------------------------------------------------------------------------
\begin{frame}
\begin{center}
Conference materials (slides, computing code, corpus, literature) can be found
using the link below: \\
\href{https://github.com/neringabr/conference-ktu}{https://github.com/neringabr/conference-ktu}
\end{center}
\end{frame}

\end{document}
